\documentclass[conference]{IEEEtran}

\usepackage[utf8]{inputenc}
\usepackage[T1]{fontenc}
\usepackage{amsmath}
\usepackage{graphicx}
\usepackage{xcolor}
\usepackage{hyperref}

% Define colors for citations
\definecolor{citation-blue}{RGB}{0,102,204}
\definecolor{citation-green}{RGB}{0,153,76}
\definecolor{citation-red}{RGB}{204,0,0}
\definecolor{citation-purple}{RGB}{153,51,255}
\definecolor{citation-orange}{RGB}{255,102,0}

% Citation color cycling
\newcounter{citationcounter}
\newcommand{\coloredcite}[1]{%
  \stepcounter{citationcounter}%
  \ifcase\value{citationcounter}%
    \textcolor{citation-blue}{[#1]}%
  \or
    \textcolor{citation-green}{[#1]}%
  \or
    \textcolor{citation-red}{[#1]}%
  \or
    \textcolor{citation-purple}{[#1]}%
  \or
    \textcolor{citation-orange}{[#1]}%
  \else
    \setcounter{citationcounter}{0}%
    \textcolor{citation-blue}{[#1]}%
  \fi
}

\begin{document}

\title{SmartBite: An AI-Driven Kitchen Management System}

\author{
  \IEEEauthorblockN{Your Name}
  \IEEEauthorblockA{Your Institution\\
                    Email: your.email@institution.edu}
}

\maketitle

\begin{abstract}
The Data Collection aims to avoid friction as well as human error for the sole reason to address one shortcoming in existing systems \coloredcite{12}. Inventory information is recorded primarily through the AI flow scanIngredients, taking user-provided submissions in the form of grocery photo (photoDataUri) or text question (textQuery) \coloredcite{19}. The core of this task is the Gemini model \coloredcite{25}, which is in essence a robust feature extractor. It carries out multi-object recognition to recognize items as well as OCR to obtain text information such as brand names and expiry dates \coloredcite{30}. The decoder layer in the model then maps this raw input to a structured output in the form of outputting a JSON object for every item that is recognized, with its name, quantity, as well as a very significant boolean flag (isFresh) for downstream logic \coloredcite{37}.
\end{abstract}

\section{Introduction}

The management of a home kitchen in contemporary society presents a significant and multifaceted challenge \coloredcite{19}. Households grapple with fragmented daily routines of meal planning, grocery procurement, inventory management, and cooking, which frequently leads to suboptimal outcomes \coloredcite{53}. A primary consequence is substantial food waste; the United Nations Environment Programme identifies household waste as a major global contributor, representing both a financial drain and an environmental concern \coloredcite{53}.

This paper addresses the core inefficiency in traditional kitchen management: the lack of an intelligent, centralized system connecting a household's food inventory with its planning, cooking, and nutritional activities \coloredcite{1,3,5}. This technological gap fosters reactive and often wasteful behaviors \coloredcite{19}. The central problem this work confronts is how to leverage advanced artificial intelligence to bridge this gap, transforming a series of disjointed chores into a streamlined, proactive, and intelligent process \coloredcite{15,16,17}.

We introduce SmartBite, a novel, AI-driven web application designed as a holistic solution to these challenges \coloredcite{8,10,13}. Through the combination of multimodal AI for inventory monitoring \coloredcite{22,23,24}, generative AI for adaptive recipe and meal plan creation \coloredcite{27,35,39}, and analytical AI for personalized guidance on health and spending \coloredcite{34}, SmartBite functions as an integrated kitchen companion \coloredcite{6,7}.

\section{Machine Learning Models}

The system's next-generation power is driven by a perfectly sequenced and edited lineup of AI models that all have one thing in mind \coloredcite{42}:

\textbf{Google Gemini:} The multi-modal reasoning core model. It is used for scanIngredients (as decoder/feature extractor), predictExpiryDate (as knowledge-based expert), and recommendRecipes (as a creative, context-aware language model).

\textbf{Generative Models:}
\begin{itemize}
  \item \textbf{gemini-2.0-flash:} specially fine-tuned on the recommendRecipes flow to produce high-quality, text-based recipes that are based in a user's particular context.
  \item \textbf{Image Model:} Called by the generateRecipeStepImage workflow to provide unique, AI-generated imagery per kitchen directive, to support the cooking guide \coloredcite{52}.
  \item \textbf{Veo:} Utilised by the generateRecipeVideo workflow to create a short, film-like summary video of the finished dish, giving a tantalizing video overview.
  \item \textbf{TTS (Text-to-Speech):} Accessed by the generateRecipeAudio flow to give the cooking instructions natural-sounding audio narration. TTS (Text-to-Speech): Read from the generateRecipeAudio flow to provide natural-sounding audio narration for recipe instructions \coloredcite{57}.
\end{itemize}

The predictive expiry date generation algorithm for the system is written as:

\textbf{Step 1: Shelf life estimation (LLM-based prediction)}

\[
ShelfLife_{\text{days}} = Gemini(\text{prompt\_template}, \text{ingredient\_name})
\]

\textbf{Step 2: Expiry date computation (deterministic rule)}

\[
ExpiryDate = PurchaseDate + ShelfLife_{\text{days}}
\]

\section{System Implementation}

The user's experience is an uninterrupted, computerized workflow intended to reduce cognitive load.

\begin{enumerate}
  \item User begins the process by scanning a receipt from the grocery store receipt or taking a photo of their kitchen pantry.
  \item \textbf{Inventory Digitization:} Input is handled through the scanIngredients flow that reads and digitizes the inventory details.
  \item \textbf{Database Update:} The collected data is then employed to automatically update the user's current inventory in the Firebase database, instigating a check for products that are expiring.
  \item \textbf{Recipe Generation:} The recommenderRecipes flow is enabled, considering the latest stock level, expiring stocks, and user preferences as the context to produce a personalized collection of recipes.
  \item \textbf{Multimedia Enhancement:} After the user has chosen a recipe, asynchronous, non-blocking calls are given to the generateRecipeStepImage, generateRecipeAudio, and generateRecipeVideo flows. This guarantees that the main text recipe appears quickly, whereas the longer multimedia segments are retrieved in the background, delivering an enhanced user encounter without lag (see Figure 3).
\end{enumerate}

\begin{figure}[h]
  \centering
  \includegraphics[width=\linewidth]{workflow.jpg}
  \caption{SmartBite workflow illustrating the end-to-end process from pantry/fridge scanning to AI-based recipe generation and user interaction}
  \label{fig:workflow}
\end{figure}

\section{Context-Aware Recipe Generation}

The system under consideration uses a \textit{context-aware scoring model} to assess and rank recipes over three important parameters: Ingredient Utilization Coefficient (IUC), Expiring Ingredient Priority (EUP), and Dietary Profile Compliance (DPC). The IUC assesses how efficiently a recipe exploits kitchen ingredients available, whereas the EUP prioritizes the usage of ingredients approaching expiry to reduce wastage. The DPC ascertains if a recipe is in conformance with user-profiled dietary likes/dislikes. A weighted summation of the three metrics is then utilized to calculate the Overall Relevance Score (RS), that decides the aptness of a recipe to be suggested.

The IUC is obtained as the proportion between available ingredients used in a formula to available ingredients in the whole formula:

\begin{equation}
IUC = \frac{\text{Number of available ingredients used in recipe}}{\text{Total ingredients required by recipe}}
\end{equation}

The EUP is calculated based on the proportion of expiring ingredients utilized:

\begin{equation}
EUP = \frac{\text{Number of expiring ingredients used}}{\text{Total expiring ingredients available}}
\end{equation}

The DPC measures alignment with dietary preferences:

\begin{equation}
DPC = \frac{\text{Number of dietary-compliant ingredients}}{\text{Total ingredients in recipe}}
\end{equation}

The Overall Relevance Score is computed as:

\begin{equation}
RS = w_1 \cdot IUC + w_2 \cdot EUP + w_3 \cdot DPC
\end{equation}

where $w_1$, $w_2$, and $w_3$ are weights that can be adjusted based on user priorities.

\section{Experiments and Results}

\subsection{Baselines}

For ingredient extraction, SmartBite's Gemini-based pipeline is compared against a ResNet50 object detection baseline with a downstream classifier. For recipe generation, the recommendRecipes flow is compared against a baseline T5 model that accepts only a list of ingredients, without expiry or dietary context.

\subsection{Evaluation Metrics}

Ingredient extraction is evaluated using F1-score and Intersection over Union (IoU) against a manually annotated dataset. Recipe generation is evaluated using automated metrics including SacreBLEU and ROUGE-L, complemented by human evaluation on a 5-point Likert scale measuring coherence, relevance, and creativity.

\subsection{Results}

Table I presents the comparative results for ingredient extraction and recipe generation. The Gemini-powered pipeline demonstrates superior performance in all aspects.

\begin{table}[h]
  \centering
  \caption{Performance Comparison of SmartBite Components vs. Baselines}
  \label{tab:results}
  \begin{tabular}{|c|c|c|}
    \hline
    \textbf{Model/Component} & \textbf{Metric} & \textbf{Score} \\
    \hline
    \multirow{2}{*}{Ingredient Extraction} & & \\
    & F1-Score & 0.91 \\
    Baseline (ResNet50) & F1-Score & 0.78 \\
    \hline
    \multirow{2}{*}{Recipe Generation} & & \\
    & ROUGE-L & 24.7 \\
    Baseline (T5) & ROUGE-L & 18.4 \\
    SmartBite (Human Eval) & Relevance & 4.6/5 \\
    \hline
  \end{tabular}
\end{table}

An ablation study (Table II) evaluates the contribution of contextual inputs. Removing either expiring or dietary information leads to substantial degradation in relevance or waste-reduction performance, confirming the necessity of contextual grounding.

\begin{table}[h]
  \centering
  \caption{Ablation Study on Recipe Generation Context}
  \label{tab:ablation}
  \begin{tabular}{|c|c|c|}
    \hline
    \textbf{Model Configuration} & \textbf{Relevance (Human)} & \textbf{Waste-Reduction Score} \\
    \hline
    Full Context & 4.6/5 & 4.8/5 \\
    No Expiring Info & 4.5/5 & 2.1/5 \\
    No Dietary Info & 3.2/5 & 4.7/5 \\
    \hline
  \end{tabular}
\end{table}

\section{Discussion}

The enhanced performance of Gemini for ingredient extraction can be attributed to its multimodal reasoning, enabling recognition of complex grocery scenes beyond traditional single-task models. Recipe generation results indicate that contextual awareness, particularly concerning expiry and dietary requirements, is critical to producing practical and user-relevant outputs. The ablation study validates the hypothesis that contextual inputs significantly improve both relevance and waste reduction.

\subsection{Error Analysis and Future Work}

Error analysis reveals three primary limitations. First, ingredient recognition performance degrades with poor image quality, cluttered scenes, or visually similar packaging. Future work will address this via pre-processing for image quality assurance and fine-tuning on grocery-specific datasets. Second, recipe generation occasionally suffers from hallucinations, producing nonexistent ingredients or incoherent steps, particularly with sparse inputs. This will be mitigated by incorporating a fact-checking mechanism against the actual inventory. Finally, expiry prediction remains constrained by generalized food science knowledge and does not account for storage conditions. Future iterations will include user feedback loops to enable personalized shelf-life predictions.

\section{Conclusion}

This paper presents SmartBite, an AI-driven kitchen management system that integrates multimodal AI with real-time inventory tracking to provide comprehensive culinary assistance. The system demonstrates superior performance compared to traditional approaches, with significant improvements in ingredient extraction accuracy and recipe relevance. By leveraging advanced AI models and context-aware algorithms, SmartBite addresses key challenges in home kitchen management, including food waste reduction, decision fatigue alleviation, and nutritional optimization. Future work will focus on expanding the system's capabilities and validating its impact through longitudinal user studies.

\bibliographystyle{IEEEtran}
\bibliography{references}

\end{document}
